% !TeX root = ../build/main.tex

Vocdoni Z ensures user anonymity by anonymizing both the ballot and the voter's identity, employing cryptographic techniques that maintain privacy even in the face of future quantum computing threats. The system uses the \textbf{ElGamal Homomorphic Encryption scheme} to encrypt ballots, allowing for the aggregation of votes without revealing individual choices. Since encrypted ballots are stored in public repositories like Ethereum blobs—which, although removed after some weeks, may still be accessible—it is crucial to prevent any association between decrypted ballots and voter identities. By decoupling the voter's identity from their encrypted ballot through the use of secrets and cryptographic hashes, even if an adversary decrypts the ballots in the future, they cannot link them back to individual voters.

The identity anonymization acts as a \textbf{double security factor}, enhancing long-term privacy. While the Sequencer processing the vote can identify the voter (since voters submit proofs of eligibility and commitments), there is no incentive for the Sequencer to make this information public, and the Sequencer cannot decrypt the voter's ballot because they do not possess the private decryption keys. This ensures that voter choices remain confidential.

We have adopted this partial identity anonymization because generating fully anonymous proofs using zkSNARKs directly from digital signatures (e.g., ECDSA/EdDSA or RSA) is computationally intensive for client-side devices like browsers and smartphones. Our priority is to support a wide range of devices, making the system accessible to as many voters as possible. In the future, as cryptographic technology advances and client devices become more powerful, we anticipate being able to generate such zero-knowledge proofs efficiently on the client side. This would enable us to fully anonymize the client's identity in addition to the ballot, further enhancing user privacy without compromising accessibility or user experience.