% !TeX root = ../build/main.tex


\martai{This section is old text, it needs to be reviewed.}

\subsection{Security discussion}
\label{sec:analysis:security}

Make a section called \textit{properties}?

\subsubsection{Receipt-freeness}

In democratic processes, the ability of voters to cast their votes freely, without undue influence or coercion, is crucial. A critical aspect of maintaining this freedom is ensuring receipt-freeness, preventing voters from being able to prove to third parties how they voted. \davinci implements receipt-freeness by leveraging the properties of the ElGamal cryptosystem, and zkSNARKs.

\paragraph{Ballor re-encryption.}

The ElGamal cryptosystem over elliptic curves supports additive homomorphism, allowing operations to be performed on ciphertexts that translate to addition on the underlying plaintexts. Re-encryption is a process that refreshes the randomness of a ciphertext without changing the underlying plaintext message. This makes computationally infeasible to link the original and re-encrypted ciphertexts.

\paragraph{Handling receipts.}

To prevent voters from being able to prove how they voted, \davinci employs re-encryption of ballots by the Sequencers. When a voter submits an encrypted ballot, the Sequencer re-encrypts it before storing it in the state Merkle tree.

This way, the system ensures that voters cannot produce a receipt of their vote by revealing r since the stored ciphertext no longer corresponds to r. This prevents vote-buying and coercion by third parties.

\paragraph{Handling collusion.}

To further enhance receipt-freeness and mitigate the risk of collusion between voters and Sequencers, \davinci allows voters to overwrite their votes. A voter can submit multiple votes, with each new submission replacing the previous one.

When a Sequencer receives a new vote from a voter who has already voted, it performs the following steps:

\begin{enumerate}
	\item \textbf{Detect overwrite}: The Sequencer checks the state Merkle tree using the voter's nullifier to determine if the voter has previously cast a vote.
	\item \textbf{Subtract previous vote}: The Sequencer takes the existing encrypted ballot and adds it to a "Subtractive Results" accumulator. This effectively negates the previous vote in the final tally.
	\item \textbf{Add new vote}: The new encrypted ballot is added to the "Results" accumulator.
	\item \textbf{Update state Merkle tree}: The Sequencer re-encrypts the new ballot and updates the state Merkle tree with this re-encrypted ballot.
\end{enumerate}

\paragraph{Concealing vote overwrites.}

To prevent observers from detecting when overwrites occur, the Sequencers regularly re-encrypt a random subset of ballots in the state Merkle tree during each state update. This process obscures the occurrence of overwrites, as re-encryptions are indistinguishable from standard re-randomizations performed for privacy enhancement.

By re-encrypting random ballots, the system increases the entropy and makes it statistically improbable for an adversary to determine if a specific ballot was overwritten or simply re-randomized.

\subsubsection{Privacy}

\Davinci ensures user anonymity by anonymizing both the ballot and the voter's identity, employing cryptographic techniques that maintain privacy even in the face of future quantum computing threats. The system uses the ElGamal homomorphic encryption scheme to encrypt ballots, allowing for the aggregation of votes without revealing individual choices. Since encrypted ballots are stored in public repositories like Ethereum blobs—which, although removed after some weeks, may still be accessible—it is crucial to prevent any association between decrypted ballots and voter identities. By decoupling the voter's identity from their encrypted ballot through the use of secrets and cryptographic hashes, even if an adversary decrypts the ballots in the future, they cannot link them back to individual voters.

The identity anonymization acts as a double security factor, enhancing long-term privacy. While the Sequencer processing the vote can identify the voter (since voters submit proofs of eligibility and commitments), there is no incentive for the Sequencer to make this information public, and the Sequencer cannot decrypt the voter's ballot because they do not possess the private decryption keys. This ensures that voter choices remain confidential.

We have adopted this partial identity anonymization because generating fully anonymous proofs using zkSNARKs directly from digital signatures (e.g., ECDSA/EdDSA or RSA) is computationally intensive for client-side devices like browsers and smartphones. Our priority is to support a wide range of devices, making the system accessible to as many voters as possible. In the future, as cryptographic technology advances and client devices become more powerful, we anticipate being able to generate such zero-knowledge proofs efficiently on the client side. This would enable us to fully anonymize the client's identity in addition to the ballot, further enhancing user privacy without compromising accessibility or user experience.

\subsubsection{Quantum resistance}

As quantum computing technology advances, it poses significant challenges to classical cryptographic schemes that underpin the security of digital systems, including voting platforms like \davinci. Ensuring that the system remains secure in the face of quantum threats is crucial for its longevity and trustworthiness.

Quantum computers have the potential to solve certain mathematical problems exponentially faster than classical computers. Notably, Shor's algorithm allows quantum computers to efficiently factor large integers and compute discrete logarithms, undermining the security of widely used cryptographic schemes such as RSA, DSA, ECDSA, and the ElGamal cryptosystem.

However, the design of \davinci incorporates mechanisms that preserve voter anonymity even in the face of future quantum attacks:

\begin{itemize}
	\item Detachment of identity and encrypted Ballot: The voter's identity and their encrypted ballot are decoupled through the use of a secret in the nullifier. The nullifier is computed as:
	$$ N = \text{Hash}(\processid \conc s) $$	
	where is a secret known only to the voter. This means that even if an adversary decrypts the encrypted ballots using a quantum computer, they cannot link a decrypted vote back to a voter's identity in the census without knowledge of the secret $s$.
	\martai{Instead of ``detachment", I propose ``unlinkability": identity unlinkability.}
	\item Quantum-resistant hash functions: the \nullifier and \commitment are computed using cryptographic hash functions that are believed to be resistant to quantum attacks (e.g., SHA-3). While Grover's algorithm can provide a quadratic speedup in searching for preimages, using sufficiently long hash outputs (e.g., 256 bits) mitigates this risk.
\end{itemize}

To further safeguard against quantum threats, the following measures can be implemented in the future:

\begin{enumerate}
	\item Adopt post-quantum signature schemes. Replace ECDSA/EdDSA with quantum-resistant algorithms such as CRYSTALS-Dilithium, Falcon, or Rainbow, which are based on hard lattice problems.
	
	\item Explore lattice-based homomorphic encryption schemes. Replace ElGamal cryptosystem with quantum-resistant alternatives such as the Brakerski-Gentry-Vaikuntanathan (BGV) or the Brakerski/Fan-Vercauteren (BFV) schemes.
	
	\item Adopt quantum-resistant zero-knowledge proof systems, such as zkSNARK constructions based on post-quantum assumptions or \textbf{zkSTARKs}.
\end{enumerate}

\martai{Citations to protocols are missing.}

\martai{If it is not quantum resistant, I would leave it to future work. Although unconventional, we could also make a ``protocol weaknesses'' section.}

\subsubsection{Data availability}

\subsection{Implementation}
\label{sec:analysis:implementation}

\martai{Link to repos, etc. + details such as circuits have been implemented using \texttt{circom}, this and that.}

\paragraph{Circuits.}
\begin{itemize}
	\item Circuit 1: circom/snarkJS, $\sim 53.000$ constraints.
	\item Circuit 2: gnark, $\sim 3.1$ million constraints.
	\item Circuit 3: gnark, 40.000 $\times$ (number of votes) constraints.
	\item Circuit 4: gnark, $\sim 4$ million constraints.
\end{itemize}

\subsection{Performance evaluation}
\label{sec:analysis:performance-evaluation}

\martai{E.g. circuit constraints benchmarks.}
