% !TeX root = ../build/main.tex

Online voting remains one of the most studied yet elusive applications in applied cryptography. As digital services expand across both public and private sectors, secure and universally verifiable online voting systems have become an increasingly desirable goal. Remote electronic voting promises scalability, accessibility, and administrative efficiency; yet the design of systems that simultaneously ensure privacy, integrity, coercion resistance, and transparency under realistic adversarial models remains a formidable challenge. Numerous deployments—including iVote, Helios, and the Estonian I‑voting system—have revealed deep-rooted usability flaws and security gaps, particularly when verifiability mechanisms are either poorly understood by users or reliant on centralized infrastructure.

Against this backdrop, the Vocdoni project was initiated in 2018 with the aim of rethinking online voting from first principles. The name Voĉdoni, meaning “to give voice” in Esperanto, reflects the project’s foundational goal: to empower collectives—from small associations to millions of citizens—to engage in secure and verifiable decision-making, regardless of technological or institutional barriers. Central to this vision was the idea that voting is not limited to formal governmental elections but serves as a more general-purpose mechanism for collective signaling. Vocdoni introduced a fully anonymous, end-to-end verifiable voting system designed to operate efficiently across a range of devices, including smartphones. To support these goals, the team deployed a custom infrastructure emphasizing resilience, neutrality, and transparency.

Technically, the architecture of Vocdoni was grounded on a bespoke Byzantine Fault Tolerant (BFT) layer-1 blockchain, named Vochain. At the time, zero-knowledge proof systems such as zkSNARKs were emerging but not yet practical for most deployments. Vochain provided a performant and low-cost environment (achieving approximately 700 transactions per second) where advanced cryptographic tools could be used without the constraints imposed by EVM-based blockchains. The ability to issue voting transactions without requiring user fees enabled broader accessibility. Over several years of development and deployment, this architecture proved both viable and valuable in practice. However, broader adoption as a universal voting protocol highlighted the need for further architectural refinements and stronger formal guarantees.

In this work, we introduce \davinci, a new protocol that builds upon the lessons and conceptual groundwork laid by Vocdoni. DaVinci adopts a modular design and integrates state-of-the-art cryptographic tools—including succinct zero-knowledge proofs, improved bulletin board constructions, and robust coercion resistance mechanisms—reflecting the most recent advances in academic research. Unlike monolithic designs, DaVinci is conceived as a composable primitive: a foundational layer intended to support secure and verifiable voting in diverse contexts, from blockchain governance to institutional elections. This shift in design philosophy aims to address long-standing challenges in the field, offering a cleaner abstraction with clearer security boundaries and formal underpinnings.

\subsection{Contributions}
\label{sec:introduction:contributions}

\begin{itemize}
	\item The Vocdoni voting protocol.
	\item The Vocdoni ballot protocol.
\end{itemize}

\subsection{Related work}
\label{sec:introduction:sota}

\begin{itemize}
	\item State of the art of e-voting: \url{https://azkr.org/blog/evoting-review/}.
\end{itemize}

\subsection{Paper organization}
\label{sec:introduction:organization}

\begin{itemize}
	\item Section~\ref{sec:background}: background.
	\item Section~\ref{sec:protocol-intuition}: high overview of the election process. We omit technical details.
	\item Section~\ref{sec:cryptographic-primitives}: cryptographic primitives used in the voting process. A description of the primitives but also the details of their instantiation, that is, the specific protocols being used.
	\item Section~\ref{sec:vocdoni-protocol}: description of the voting process.
	\item Section~\ref{sec:ballot-protocol}: a proposed protocol for ballots. The rules of this protocol are implemented in the process management smart contracts.
	\item Section~\ref{sec:token}: the Vocdoni token.
	\item Section~\ref{sec:analysis}: analysis of the proposed protocols. It includes a security discussion of the properties of the protocol, implementation details, and a performance evaluation.
	\item Section~\ref{sec:conclusions}: we finish with some last remarks.
\end{itemize}