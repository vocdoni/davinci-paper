% !TeX root = ../build/main.tex

\subsection{Contributions}
\label{sec:introduction:contributions}

\begin{itemize}
	\item The Vocdoni voting protocol.
	\item The Vocdoni ballot protocol.
\end{itemize}

\subsection{Related work}
\label{sec:introduction:sota}

\begin{itemize}
	\item State of the art of e-voting: \url{https://azkr.org/blog/evoting-review/}.
\end{itemize}

\subsection{Paper organization}
\label{sec:introduction:organization}

\begin{itemize}
	\item Section~\ref{sec:background}: background.
	\item Section~\ref{sec:protocol-intuition}: high overview of the election process. We omit technical details.
	\item Section~\ref{sec:cryptographic-primitives}: cryptographic primitives used in the voting process. A description of the primitives but also the details of their instantiation, that is, the specific protocols being used.
	\item Section~\ref{sec:vocdoni-protocol}: description of the voting process.
	\item Section~\ref{sec:ballot-protocol}: a proposed protocol for ballots. The rules of this protocol are implemented in the process management smart contracts.
	\item Section~\ref{sec:token}: the Vocdoni token.
	\item Section~\ref{sec:analysis}: analysis of the proposed protocols. It includes a security discussion of the properties of the protocol, implementation details, and a performance evaluation.
	\item Section~\ref{sec:conclusions}: we finish with some last remarks.
\end{itemize}